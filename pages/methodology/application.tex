This chapter introduces the web application test, which demonstrates the possibility of testing the entire web application within the limitations aforementioned presented in \autoref{sec:limits}.



This chapter is about detecting security issues in a web application using expoSE. It first describes the general idea, then shows examples of security vulnerabilities, mostly XSS vulnerabilities. 

\section{XSS}

Cross-site scripting, also known as XSS, is the practice of executing malicious code on a remote server or another client by exploiting vulnerabilities in the server-side input validation. 


\begin{tikzpicture}[node distance=2cm]


% Define styles for the nodes and arrows

\tikzstyle{startstop} = [rectangle, rounded corners, minimum width=3cm, minimum height=1cm,text centered, draw=black, fill=red!30]

\tikzstyle{process} = [rectangle, minimum width=3cm, minimum height=1cm,text centered, draw=black, fill=orange!30]

\tikzstyle{decision} = [diamond, minimum width=3cm, minimum height=1cm, text centered, draw=black, fill=green!30]

\tikzstyle{arrow} = [thick,->,>=stealth]

a
% Nodes

\node (start) [startstop] {Malicious Attacker Client};

\node (pro1) [process, below of=start] {\textless script\textgreater alert("Hello")\textless/script\textgreater};

\node (dec1) [decision, below of=pro1, yshift=-1cm] {Server};

\node (pro2a) [startstop, below of=dec1, xshift=-3cm] {Unassuming Client};

\node (pro2b) [startstop, below of=dec1, xshift=3cm] {Unassuming Client};

\node (stopA) [process, below of=pro2a] {Hello};
\node (stopB) [process, below of=pro2b] {Hello};


% Arrows

\draw  [arrow](start) -- node[anchor=east] {post message} (pro1);

\draw  [arrow](pro1) -- node[anchor=east] {store message}(dec1);

\draw  [arrow](dec1) -- node[anchor=east] {distribute message} (pro2a);

\draw [arrow] (dec1) -- node[anchor=west] {distribute message} (pro2b);

\draw [arrow] (pro2a) -- node[anchor=west] {execute code}(stopA);
\draw [arrow] (pro2b) -- node[anchor=west] {execute code}(stopB);



\end{tikzpicture}


\section{Initial Test Plain JS Application}

At the beginning of this endeavour, we built a small server in plain JavaScript Application, in order to see, whether it is even possible, to generate correct requests.

\subsection{Idea}

The idea was to create an actual plain JavaScript server, using the http package, and then strip it of everything, that relies on http, basically bypassing the actual server - client connection. 
\subsection{Test evaluation}
\section{Express Application}
\subsection{Server}
\subsection{Client}
\subsection{Necessary Modifications}
