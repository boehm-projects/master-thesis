This chapter presents how to use the express model and how to test a web application. 

\section{Necessary Modifications} 
\label{sec:nec-mods}

Since the model is located in expoSE right now, all it requires is a slight modification to the imports and how express is called.
This means, the import requires the relative path, and cannot be imported via the node package manager (NPM). 
The usage is also slightly different, as the express model instance needs to be initiated after importing, while express itself is initiated when it is imported. 
Other than that, it can be used just like express.js, as the basic methods are implemented.





\section{Testing} 
\label{sec:app-testing}
To carry out the end-to-end testing of a web application, we require a client that has the following properties:\\ 
First, it must utilize the S\$ library, which serves as the interface to ExpoSE, as explained in \autoref{sec:expose}. 
Subsequently, all variables within the request should be substituted with symbolic inputs. 
In instances where the server expects a complex data structure in the body of the request,
it is advisable to send an object encapsulating all fields that the server can process across all routes.
This recommendation stems from the challenges associated with symbolic objects, as discussed in \autoref{sec:sym-obj}. 
This approach allows for the generalization of a single test without necessitating multiple executions to cover all required fields,
as any fields not utilized by the specific route are ignored. 
A client such as  \autoref{lst:client-example} could be used to generate the requests.

\begin{lstlisting}[language=JavaScript, float, caption={[Example Client]An Example Client for symbolic requests}, label={lst:client-example}]
const S$ = require("S$");
const { Request } = require("../request");
const { Response } = require("../response");
import HttpMethods from "../express_model/http_methods.enum";


generateRequest (){
    const req  = new Request(
        S$.symbol("path", "/"),
        HttpMethods[S$.symbol("method", 0)],
        {
            "field": S$.symbol("data", "")
        }
    )
    return req;
}
\end{lstlisting}
As explained in \autoref{sec:init-test-plain}, we use our mock implementation for the request and response objects. In the \lstinline{generateRequest} function, we build the symbolic request, which has at least 2 parameters, a URL, and an HTTP method. For the URL parameter, we create a symbolic variable \lstinline{path}, with the seed input \lstinline{"/"}, eliminating all variable types but string. 
We then pass an array containing all http methods, accessing the element via the symbolic index variable \lstinline{method}, with a starting seed like before, to indicate it being an integer.
Additionally, we create an object containing a symbolic field, which corresponds to an expected field in the server. As previously mentioned, we do not recommend using a symbolic object.


Testing a web application using the model is a straightforward process.
By injecting the client into the server code and invoking the \textsc{listen}
function  that typically converts an input into an HTTP request, \lstinline{let response = server.listen('8080,callback, << SYMBOLIC } \par\noindent\lstinline{CLIENT REQUEST >> );}, it becomes feasible to generate inputs that test all routes, incorporating various parameters and bodies.


Once the client is either modified or created to produce a symbolic request, 
a singular execution of ExpoSE should suffice to generate diverse requests that effectively cover all possible paths of a request and trigger all potential unique responses, as long as it is a distinct and reachable path.  

Of course, there also exists the option to unit test the server functions without any need for a client and the express model. However, this does not leverage the power of symbolic execution, as it only generates inputs for a function, instead of the whole system, requiring, albeit few, manual generation of tests for each function.

