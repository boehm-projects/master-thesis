This chapter introduces the web application test, which demonstrates the possibility of testing the entire web application within the limitations aforementioned presented in \autoref{sec:limits}.



This chapter is about detecting security issues in an web application using expoSE. It first describes the general idea, then shows examples of security vulnerabilities, mostly XSS vulnerabilities. 

\section{XSS}

\begin{tikzpicture}[node distance=2cm]


% Define styles for the nodes and arrows

\tikzstyle{startstop} = [rectangle, rounded corners, minimum width=3cm, minimum height=1cm,text centered, draw=black, fill=red!30]

\tikzstyle{process} = [rectangle, minimum width=3cm, minimum height=1cm,text centered, draw=black, fill=orange!30]

\tikzstyle{decision} = [diamond, minimum width=3cm, minimum height=1cm, text centered, draw=black, fill=green!30]

\tikzstyle{arrow} = [thick,->,>=stealth]


% Nodes

\node (start) [startstop] {Start};

\node (pro1) [process, below of=start] {Process 1};

\node (dec1) [decision, below of=pro1, yshift=-1cm] {Decision 1};

\node (pro2a) [process, below of=dec1, xshift=-3cm] {Process 2A};

\node (pro2b) [process, below of=dec1, xshift=3cm] {Process 2B};

\node (stop) [startstop, below of=pro2a] {Stop};


% Arrows

\draw  (start) -- (pro1);

\draw  (pro1) -- (dec1);

\draw  (dec1) -- node[anchor=east] {Yes} (pro2a);

\draw  (dec1) -- node[anchor=west] {No} (pro2b);

\draw [arrow] (pro2a) -- (stop);


\end{tikzpicture}


\section{Initial Test Plain JS Application}
\subsection{Idea}
\subsection{Test evaluation}
\section{Express Application}
\subsection{Server}
\subsection{Client}
\subsection{Necessary Modifications}
