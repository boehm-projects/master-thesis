\documentclass[seminar]{plai}

\usepackage{tabularx}

\usepackage{lipsum}

\title{Learning Opportunities in University Seminars}
\seminar{Proseminar Malware-Analyse}
\author{Erika Mustermann}
\matrikelnummer{12345678}
\semester{WiSe 2023/24}
\email{erika.mustermann@campus.lmu.de}


%%%%%%%%%%%%%%%%%%%%%%%%%%%%%%%%%%%%%%%%%%
\begin{document}

\maketitle

\begin{abstract}
    \lipsum[1]
\end{abstract}


%%%%%%%%%%%%%%%%%%%%%%%%%%%%%%%%%%%%%%%%%%
\section{Introduction}
\label{sec:introduction}

The topic at hand is an important problem discussed several times in the recent literature~\cite{oakland23-xfl,spmag23-mlmalware,cryptosec11}. Industry requires a solution to this problem, as it is a major bottleneck in the development of new products~\cite{statemerging-patent}. As \citet{conformance-testing-arxiv} point out, it also is a major challenge in the field of conformance testing. 
\lipsum[3-8]

In particular, this article makes the following contributions:
\begin{itemize}
    \item We define the problem of abc as an instance of xyz~(\autoref{sec:phrasing}).
    \item We review the literature on xyz and show how corresponding solutions can be transferred to abc, following the example of def~(\autoref{sec:transfer}).
    \item We demonstrate how to encode a specific example of abc to WizWoz, an implementation of def~(\autoref{sec:remaining}), and confirm that it effectively solves the problem~(\autoref{sec:evaluation}).
\end{itemize}

%%%%%%%%%%%%%%%%%%%%%%%%%%%%%%%%%%%%%%%%%%
\section{Background}
\label{sec:background}

We begin by introducing the background on university seminars~(\autoref{sec:seminars}), before reviewing the state of the art in measuring learning outcomes~(\autoref{sec:learning-outcomes}).

\subsection{University Seminars}
\label{sec:seminars}

\lipsum[1-3]

\subsection{Measuring Learning Outcomes}
\label{sec:learning-outcomes}

\lipsum[1-3]

%%%%%%%%%%%%%%%%%%%%%%%%%%%%%%%%%%%%%%%%%%
\section{Overview}
\label{sec:overview}

\lipsum[1-3]


%%%%%%%%%%%%%%%%%%%%%%%%%%%%%%%%%%%%%%%%%%
\section{Main Contents}
\label{sec:phrasing}

\lipsum[1-3]

\begin{figure}[t]
\begin{lstlisting}[language=Python]
def foo():
    return 42   # the answer to everything

def bar():
    return f'The answer is: {foo()}'
\end{lstlisting}
\caption{A simple example of a program. Figure captions go below.}
\label{fig:example-program}
\end{figure}

\autoref{fig:example-program} shows a simple example of a program. It is easy to see that the program is correct, as it returns the correct result.


%%%%%%%%%%%%%%%%%%%%%%%%%%%%%%%%%%%%%%%%%%
\section{More Main Contents}
\label{sec:transfer}

\lipsum[1-3]


%%%%%%%%%%%%%%%%%%%%%%%%%%%%%%%%%%%%%%%%%%
\section{Remaining Main Contents}
\label{sec:remaining}

\lipsum[1-3]


%%%%%%%%%%%%%%%%%%%%%%%%%%%%%%%%%%%%%%%%%%
\section{Evaluation}
\label{sec:evaluation}

\lipsum[1-3]

\autoref{tab:results} shows the results of the evaluation. As can be seen, Method 1 outperforms Method 2. This is in line with the results of \citet{phdthesis-kinder}, who also found that Method 1 is superior to Method 2. 

\begin{table}[t]
    \centering
    \caption{Results of the evaluation. Table captions go above.}
    \label{tab:results}
    \begin{tabularx}{.7\linewidth}{Xrr}
        \toprule
        \textbf{Method} & \textbf{Precision} & \textbf{Recall} \\
        \midrule
        Method 1 & 0.42 & 0.23 \\
        Method 2 & 0.23 & 0.42 \\
        \bottomrule
    \end{tabularx}
\end{table}

\lipsum[1-3]


%%%%%%%%%%%%%%%%%%%%%%%%%%%%%%%%%%%%%%%%%%
\section{Related Work}
\label{sec:related-work}

\lipsum[1-3]


%%%%%%%%%%%%%%%%%%%%%%%%%%%%%%%%%%%%%%%%%%
\section{Conclusion}
\label{sec:conclusion}

\lipsum[1]

%%%%%%%%%%%%%%%%%%%%%%%%%%%%%%%%%%%%%%%%%%
\bibliographystyle{plainnat}
\bibliography{bibliography}

\end{document}